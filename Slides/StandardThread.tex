%
% Slides on C++ standard threads.
%

\documentclass[landscape]{slides}
\usepackage{amsmath}
\usepackage{amssymb}
\usepackage{amstext}
\usepackage{color}
\usepackage{fancyvrb}
\usepackage[pdftex]{graphicx}
\usepackage{listings}
\usepackage{times}
\usepackage{url}

%
% For landscape mode slides:
%  1) use \documentclass[landscape]{slides}
%  2) latex main
%  2) dvips -tlandscape main -o 
%
% You can also try
%  1) use \documentclass[landscape]{slides}
%  2) latex main
%  3) texdvipdfm -l main.dvi
%

\setlength{\topmargin}{-0.75in}
\setlength{\textheight}{7in}

% colour defs
\input colordvi
%\def\Red#1{\Color{0 0.70 0.70 0.2}{#1}}
%\def\Green#1{\Color{0.70 0 0.5 0.2}{#1}}
%\def\Blue#1{\Color{0.70 0 0 0.2}{#1}}
\definecolor{Background}{rgb}{1.00, 1.00, 1.00}
\definecolor{Beige}     {rgb}{0.96, 0.96, 0.86}
\definecolor{Black}     {rgb}{0.00, 0.00, 0.00}
\definecolor{Blue}      {rgb}{0.00, 0.00, 1.00}
\definecolor{DarkGold}  {rgb}{0.30, 0.28, 0.00}
\definecolor{emphcolor} {rgb}{0.40, 0.20, 0.60}
\definecolor{Gold}      {rgb}{1.00, 0.84, 0.00}
\definecolor{headcolor} {rgb}{0.55, 0.00, 0.00}
\definecolor{itemcolor} {rgb}{0.10, 0.10, 0.55}
\definecolor{LightBlue} {rgb}{0.80, 0.85, 1.00}
\definecolor{Pink}      {rgb}{1.00, 0.75, 0.80}
\definecolor{refcolor}  {rgb}{0.00, 0.40, 0.40}
\definecolor{tickmarkcolor}{cmyk}{0.98, 0.30, 0.00, 0.43}
\definecolor{titlecolor}   {cmyk}{0.98, 0.30, 0.00, 0.43}
\definecolor{Yellow}    {rgb}{1.00, 0.84, 0.80}

% new tick marks for itemize
\renewcommand{\labelitemi}{ {\color{tickmarkcolor}$\bullet$} }

% new tick marks for itemize
\renewcommand{\labelenumi}{
  {\color{tickmarkcolor} \arabic{enumi}}
}

% color versions of itemize/enumerate with less spacing
\newenvironment{citemize}{
  \vspace{-0.2in}
  \begin{itemize}
    \itemsep 0in
    \color{itemcolor}
}{
\end{itemize}
}
\newenvironment{cenumerate}{
  \vspace{-0.2in}
  \begin{enumerate}[\hspace{5mm}1.]
    \itemsep 0in
    \color{itemcolor}
}{
\end{enumerate}
}

% headed slide takes 1 parameter: the heading of the slide, which will
% appear at the topright of the page
\newenvironment{hslide}[1]{
  \begin{slide}
    \begin{flushright}
      {\color{headcolor} \tiny #1}
    \end{flushright}
}{
\end{slide}
}

% titled slide takes 1 parameter: the heading of the slide, which will
% appear at the topright of the page
\newenvironment{tslide}[1]{
  \begin{slide}
    \begin{flushleft}
      {\bf \color{headcolor} #1}
    \end{flushleft}
}{
\end{slide}
}

% title - displayed as emphasised
\newcommand{\stitle}[1]{
  { \bf \color{titlecolor} #1 }
}

% comments
\newcommand{\mknote}[1]{
  \begin{note}
    \begin{flushright}
      {\it NOTES}
    \end{flushright}
    {\scriptsize #1
                
      end}
  \end{note}
}

% end web borrowed macros

\def\cemph#1{{\color{emphcolor}\emph{#1}}}
\def\ucemph#1{{\color{emphcolor}#1}}
\def\cref#1{{\color{refcolor}\small\emph{#1}}}
\def\bigref#1{{\color{refcolor}\emph{#1}}}
\def\cdesc#1{{\color{titlecolor}\small\emph{#1}}}

% beat slides class fonts into submission
% first, use most of package times
\renewcommand{\sfdefault}{phv}
\renewcommand{\rmdefault}{ptm}
\renewcommand{\ttdefault}{pcr}
% then, use most of package mathptm
\DeclareSymbolFont{operators}   {OT1}{ptmcm}{m}{n}
\DeclareSymbolFont{letters}     {OML}{ptmcm}{m}{it}
\DeclareSymbolFont{symbols}     {OMS}{pzccm}{m}{n}
\DeclareSymbolFont{largesymbols}{OMX}{psycm}{m}{n}
\DeclareSymbolFont{bold}        {OT1}{ptm}{bx}{n}
\DeclareSymbolFont{italic}      {OT1}{ptm}{m}{it}

\long\def\inlong#1{{\def\titlefont{\it}{\it #1}}}
\long\def\inlong#1{}
% slide macro abbreviations
\def\centerheader#1{\begin{center}
 {{\large\titlefont\color{titlecolor}#1}\par\vskip 1.5ex}
 \end{center}}
\def\titlefont{\bf}

\def\startslide#1{
\begin{slide}
\if !#1
 \else \centerheader{#1}
\fi
}
\def\stopslide{\end{slide}}

\newcounter{section} % make macro files load without complaining


% Default settings for code listings.
\lstset{
  language=C++,
  basicstyle=\ttfamily,
  stringstyle=\ttfamily,
  commentstyle=\ttfamily,
  xleftmargin=0.25in,
  showstringspaces=false}

\lstMakeShortInline[basicstyle=\ttfamily]!

\title{\color{titlecolor}C++ Standard Threads}
\author{
  \begin{tabular}{c}
  \\[3mm]
  \Large{Peter C. Chapin} \\[2mm]
  \normalsize{Computer Information Systems}\\[5mm]
  \includegraphics[scale=0.80]{../../LaTeX/Slides/VermontTech_stack_361.jpg}\\[16mm]
  \end{tabular}
}
\date{April 8, 2015}

\begin{document}

\color{Black}
\pagecolor{Background}

\maketitle

%%%%%

\startslide{Overview}
\begin{citemize}
\item \cemph{Why?} Widespread agreement that C++ should have standardized thread handling
  facilities.
\begin{citemize}
  \item Many programs need threads.
  \item Currently must use non-portable, OS-specific facilities.
  \item Semantics of C++ 1998 with multiple threads unclear.
  \item C++ 2011 contains thread classes roughly taken from Boost.
\end{citemize}
\end{citemize}
\stopslide

%%%%%

\startslide{Basic Example}
{\small
\begin{lstlisting}
#include <thread>

void f( int count )
{
    ...
}

int main( )
{
    // Start thread, pass argument to thread function.
    std::thread my_thread( f, 42 );

    ... // Execute here and in f at the same time.

    // Wait for thread to end.
    my_thread.join( );
    return 0;
}
\end{lstlisting}
}
\stopslide

%%%%%

\startslide{Notes On Basic Example}
\begin{citemize}
  \item Thread function requirements\ldots
  \begin{citemize}
    \item Return !void!.
    \item Can take parameters as needed (any type).
    \item Can be any callable object.
      \begin{citemize}
        \item Pointer to function.
        \item Class instance with overloaded !operator( )!.
      \end{citemize}
  \end{citemize}

  \item thread behavior\ldots
    \begin{citemize}
      \item Given callable object is \cemph{moved} (or else copied).
      \item Destructor detaches (unless thread already joined).
    \end{citemize}
\end{citemize}
\stopslide

%%%%%

\startslide{Local Variables}
Each thread has its own stack. Locals are allocated on the thread's stack; globals are shared
between threads.
\vspace{5mm}
{\small
\begin{lstlisting}
int x;  // Global data.

void f( int count )
{
    int temporary = count;  // Temporary local to the thread.
    x = temporary;          // X is shared by threads.
}

// No problem with two threads executing same function.
// Each thread has its own arguments and local variables.

std::thread thread1( f, 42 );
std::thread thread2( f, 84 );
\end{lstlisting}
}
\stopslide

%%%%%

\startslide{Callable Class Instance}
Possible to use a class instance as a callable.
\vspace{5mm}
{\small
\begin{lstlisting}
struct Worker {
    void operator( )( int count );
};

void Worker::operator( )( int count )
{
    // Do stuff...
}

int main( )
{
    // Create an instance of a callable object.
    Worker callable;
    std::thread my_thread( callable, 42 );

    ... // Execute here and in Worker::operator( ).

    my_thread.join( );
    return 0;
}
\end{lstlisting}
}
\stopslide

%%%%%

\startslide{Returning Values}
\begin{citemize}
  \item Thread callables must return !void!.
  \item Returning a value is tricky.
    \begin{citemize}
      \item The parent thread is not waiting for the result; it is busy doing something else. 
      \item Return value must be stored somewhere.
    \end{citemize}
  \item Three possibilities\ldots
    \begin{citemize}
      \item Global data (discouraged).
      \item Use a pointer parameter (okay, can be tricky).
      \item Members of a callable with class type (preferred).
    \end{citemize}
\end{citemize}
\stopslide

%%%%%

\startslide{Return Values: First Try}
This doesn't quite work.
\vspace{5mm}
{\small
\begin{lstlisting}
struct Worker {
    int result;
    void operator( )( int count );
};

void Worker::operator( )( int count ) {
    ...
    result = ...
}

int main( ) {
    Worker callable;
    std::thread my_thread( callable, 42 );
    ...
    my_thread.join( );
    // Look at callable.result here.
    return 0;
}
\end{lstlisting}
}
\stopslide

%%%%%

\startslide{Returning Values: Second Try}
Problem: The callable is \cemph{moved} (or else copied)
\vspace{5mm}
{\small
\begin{lstlisting}
#include <functional>

int main( )
{
    Worker callable;
    std::thread my_thread( std::ref( callable ), 42 );

    ...

    my_thread.join( );
    // Now callable.result is meaningful.

    return 0;
}
\end{lstlisting}
}
The !std::ref! function returns a !reference_wrapper! that can be copied into the thread object
and that forwards all access to the original callable.
\stopslide

%%%%%

\startslide{Thread Copies}
The !std::thread! class is not copyable. \cemph{What would copying a thread mean?}

But it is movable. Threads can be moved out of funtions (returned) and into ``move aware''
containers.
\vspace{5mm}
{\small
\begin{lstlisting}
std::thread make_thread( )
{
    int x;

    // Do complex stuff to prepare thread arguments.
    return std::thread( f, x );
}
...

std::vector<std::thread> my_threads;
my_threads.push_back( make_thread( ) );
\end{lstlisting}
}
It's okay to use local variable as thread parameter because it is moved (or else copied) into
the thread object.
\stopslide

%%%%%

\startslide{Exceptions}
If the callable passed to a thread thows an unhandled exception,\\
!std::terminate! is called.

\begin{citemize}
\item Normally aborts the \cemph{entire} program.
\item Probably should wrap top level thread function in a catch-all handler.
\end{citemize}
\vspace{5mm}
{\small
\begin{lstlisting}
void thread_function( )
{
  try {
    // Do stuff.
  }
  catch( ... ) {
    // Unhandled exception reached here.
    // Do something reasonable.
    // Let function end without throwing.
  }
}
\end{lstlisting}
}
\stopslide

% Thread interruption is not supported in C++ 2011. Although Boost provides a reasonably safe
% way to do it, I guess the C++ committee felt it was more complicated than it was worth. It's
% better to not try to interrupt a running without its cooperation.
%
%%%%%%%
%%
%%\startslide{Thread Interruption}
%%A thread can be interrupted.
%%
%%\begin{citemize}
%%\item Call !interrupt! method on !boost::thread! object.
%%\item Causes !boost::thread_interrupted! to be thrown in the thread.
%%  \begin{citemize}
%%    \item Thread can handle exception if it wants.
%%    \item Thread stack unwound in the usual way (local objects destroyed, etc).
%%    \item Does \cemph{not} cause termination if reaches top level.
%%  \end{citemize}
%%\item Thread can temporarily disable interruption.
%%\item Interruption can only occur at specific \textit{interruption points} in the thread.
%%\end{citemize}
%%\stopslide
%%
%%%%%%%
%%
%%\startslide{Disabling Interruption}
%%{\small
%%\begin{lstlisting}
%%void some_function( )
%%{
%%  // Constructor of di turns off interruption.
%%  boost::this_thread::disable_interruption di;
%%
%%  // Thread will not be interrupted here.
%%  // Destructor of di restores interruption status.
%%}
%%\end{lstlisting}
%%}
%%Uses constructor/destructor of special class to manage interruption.
%%
%%Restores interruption status for both normal return and if an exception is thrown.
%%\stopslide

%%%%%

\startslide{Hardware Concurrency}
Can query library to find out how many hardware threads are available.
\vspace{5mm}
{\small
\begin{lstlisting}
unsigned processor_count =
  std::thread::hardware_concurrency( );

if( processor_count == 0 ) {
  // Can't tell how many CPUs there are.
}
else {
  // Create as many threads as there are processors.
}
\end{lstlisting}
}
This is useful when doing parallel programming.
\stopslide

%%%%%

\startslide{Namespace \texttt{this\_thread}}
Functions in !std::this_thread! affect only the calling thread.

Methods of a !std::thread! object affect the thread represented by that object (if any).

A !std::thread! object might not represent a thread because\ldots
\begin{citemize}
  \item \ldots the thread has terminated.
  \item \ldots the thread has been explicitly \cemph{detached} using the !detach! method.
\end{citemize}
\stopslide

%%%%%

\startslide{Thread IDs}
Each thread has a unique ID.
\begin{citemize}
  \item Call !get_id! on the !std::thread! object.
  \item Call !std::this_thread::get_id! to get the ID of the current thread.
\end{citemize}

Can be used as keys in an associative container.
\vspace{5mm}
{\small
\begin{lstlisting}
std::map<std::thread::id, string> thread_name;

thread_name[std::this_thread::get_id( )] = "Me";
thread_name[some_thread.get_id( )] = "You";
\end{lstlisting}
}
\stopslide

%%%%%
% TODO: FIX ME! This is not the way sleeping works in C++ 2011.
\startslide{Sleeping}
The static method !std::thread::sleep! allows a thread to block until a specified time occurs.
\vspace{5mm}
{\small
\begin{lstlisting}
void f( )
{
  // Do stuff.

  // Wait until the time specified arrives.
  std::thread::sleep(
    std::get_system_time( ) +
    std::posix_time::milliseconds( 1000 ) );
}
\end{lstlisting}
}
There is also !std::this_thread_sleep!. It waits until a specified interval has elapsed.
\vspace{5mm}
{\small
\begin{lstlisting}
std::this_thread::sleep(
  std::posix_time::milliseconds( 1000 ) );
\end{lstlisting}
}
\stopslide

%%%%

\startslide{Mutual Exclusion}
C++ provides several ``mutex'' classes for locks.
\begin{citemize}
  \item !class mutex!\\
    Provides basic !lock!, !try_lock!, !unlock! methods.
  \item !class timed_mutex!\\
    Allows for specifying a time out with !timed_lock!.
  \item !class recursive_mutex!\\
    Will not deadlock if the same thread locks twice.
  \item !class recursive_timed_mutex!\\
    A recursive mutex with a !timed_lock! method.
\end{citemize}
\stopslide

%%%%%

\startslide{Using Mutex Objects}
Mutex objects can be used directly\ldots
\vspace{5mm}
{\small
\begin{lstlisting}
#include <mutex>

// Must be shared between threads (so global).
std::mutex mutual_exclusion;

void f( )
{
  mutual_exclusion.lock( );

  // This thread now has exclusive access.

  mutual_exclusion.unlock( );
}
\end{lstlisting}
}
\cemph{Not recommended}. This isn't exception safe. The !unlock! method won't be called if an
exception is thrown.
\stopslide

%%%%%

\startslide{Use Template Wrappers}
Instead use RAI enabled wrapper templates.
\vspace{5mm}
{\small
\begin{lstlisting}
std::mutex mutual_ex;

void f( )
{
  // Constructor of 'guard' locks.
  std::lock_guard<std::mutex> guard( mutual_ex );

  // This thread now has exclusive access.

  // Destructor of 'guard' unlocks.
}
\end{lstlisting}
}
\cemph{Lock is released even if an exception is thrown.}

There is also a !unique_lock! template that is similar except it can be moved (for example into
containers) and returned from functions. It has other features as well.
\stopslide

% The shared_mutex type from Boost is not part of C++ 2011 so the reader/writers example below
% would have to be done differently in C++ 2011. TODO: Show how this would be done.
%
%%%%%%%
%%
%%\startslide{An Example Using Shared Locks}
%%Solving the reader/writers problem\ldots
%%\vspace{5mm}
%%{\small
%%\begin{lstlisting}
%%using namespace boost;
%%
%%shared_mutex controller;
%%
%%void f( )
%%{
%%  shared_lock<shared_mutex> guard( controller );
%%  // Threads share access... must read only.
%%}
%%
%%void g( )
%%{
%%  unique_lock<shared_mutex> guard( controller );
%%  // This thread has exclusive access... can write.
%%}
%%\end{lstlisting}
%%}
%%\stopslide

%%%%%

\startslide{Condition Variables}
Modeled after POSIX condition variables, but with a C++ interface.
\begin{citemize}
  \item Allows a thread to \cemph{wait} until an arbitrary condition is true.
    \begin{citemize}
      \item Another thread \cemph{notifies} the waiting thread (or all waiting threads at once).
    \end{citemize}
  \item Proper use of condition variables is tricky.
    \begin{citemize}
      \item Refer to a description of POSIX condition variables to understand the issues.
    \end{citemize}
\end{citemize}
\stopslide

%%%%%

\startslide{Example}
\vspace{2mm}
{\small
\begin{lstlisting}
std::condition_variable cond;
std::mutex mut;
bool data_ready;

void wait_for_data( )
{
  std::unique_lock<std::mutex> lock( mut );
  while( !data_ready ) cond.wait( lock );
  process_data( );
}

void prepare_data( )
{
  do_the_work( );
  {
    std::lock_guard<std::mutex> lock( mut );
    data_ready = true;
  }
  cond.notify_one( );
}
\end{lstlisting}
}
\stopslide

% C++ 2011 does not support barriers. TODO: Show how this would be done.
%%%%%%%
%%
%%\startslide{Barriers}
%%A barrier allows multiple threads to wait until they have all reached a particular point.
%%\vspace{5mm}
%%{\small
%%\begin{lstlisting}
%%// Configure barrier for five threads.
%%boost::barrier sync( 5 );
%%
%%void f( )
%%{
%%  // Do work. This function is called by five threads.
%%
%%  // Wait here until all five threads reach this spot.
%%  sync.wait( );
%%}
%%\end{lstlisting}
%%}
%%Barriers are useful for parallel programming to ensure that an entire data structure has been
%%processed by a team of threads before continuing.
%%\stopslide

%%%%%

\startslide{Conclusion}
\begin{citemize}
\item C++ threads has more facilities than described here.
\item POSIX-like functionality with convenient C++ interface.
\item Boost.Thread inspired new thread features of the C++ 2011 library.
\end{citemize}
\vspace{0.5in}
\centerline{Questions?}
\stopslide

\end{document}
